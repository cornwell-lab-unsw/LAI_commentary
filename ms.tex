\documentclass[a4paper,11pt]{article}
\usepackage[osf]{mathpazo}
\usepackage{ms}
\usepackage{natbib}
\usepackage{lineno}
\usepackage{graphicx}
\usepackage{caption}
\usepackage[osf]{mathpazo}
\usepackage[T1]{fontenc}
\usepackage{textcomp}
\modulolinenumbers[5]
\linenumbers
\input{numbers}

\pdfminorversion=3

\makeatletter
\renewcommand{\@biblabel}[1]{\quad#1.}
\makeatother

\newcommand{\phyndr}{\tt phyndr}
\newcommand{\taxonlookup}{\tt taxonlookup}

\title{The difficult but important paths toward understanding amount of leaf area in forests}
\author{
William K. Cornwell$^{1,\dag}$
}

\date{}
\affiliation{
$^{1}$ Ecology and Evolution Research Centre, School of Biological, Earth and Environmental Sciences, University of New South Wales, Sydney, NSW 2052, Australia\\
$^{*}$ Email for correspondence: \texttt{wcornwell@gmail.com}\\
}

\mstype{Commentary}


\begin{document}
	
	\mstitlepage
	\parindent=1.5em
	\addtolength{\parskip}{.3em}
	\vfill
	
	\newpage
	
	\section{Introduction}
	
	While not exactly mysterious, the amount of leaf area per groud area (leaf area index, LAI)in different vegetation types and its effect on light penetration has been of long interest to ecologists.  Very early on it became clear that this was of interest to understanding the energetics of a stand of vegetation and to the regeneration of plants at the ground level.  However, despite it's clear importance, it's not an easy quantity to measure.  Blaun-Blauchet proposed a simple, semi-quantitative method to get at this.  Dobert et al. (2015) draw from this ``classic'' technique and test it against more quantitative methods.  But it was only later that a more quantitative understanding of LAI began to emerge.  
	
	In an extremely prescient paper Monsi and Saeki (1953) laid out a number of the ideas of what would become modern
	vegetation ecology, including a high quantitative understanding of leaf area index and its effect on (Note: this paper was originally published in German in the Japenese Journal of Botany and very thoughtfully translated into English and republished in Annals of Botany by Marcus Schortemeyer.)  Considering the energetics of the system as a whole, one of the key observations of Monsi and Saeki (1953) was that the dynamics at the bottom of the understory are inextricably linked to those at the top.  In the darkest of the Japenese systems they studied, the bamboo forests, the understory species were becoming light--starved and etiolated. The winners and losers in regeneration is a function at least in part of the amount leaf cover.  This is in some ways a special case of the regeneration niche sense Grubb (1977), with leaf area and light penetration playing a key role.  Thus, if a full understanding of forest dynamics rests on an understanding leaf area index.  
	
    LAI at the stand level is an emergent feature of multiple species and multiple functional groups.  Some species (e.g. Tsuga canadensis in US forests) make high LAI canopies (Ellison 2014) which means can represent a mechanism of niche construction sensu (Smee et al. 2003)

    it changes on a variety of time scales.  
	
	The amount of leaf area in an ecosystem has long been of interest to vegetation ecologists, dating at least to the beginning of the 20th century \citep{Blaun-Blauchet}.  
	
	The ecology of light penetration to the ground floor has 
	
	.  per ground area or leaf area index (LAI) is a quantity of great interest.  The earliest development of that idea that I could find is in 