\documentclass[11pt]{article}

\usepackage{amsmath, amsthm, amssymb,graphicx,natbib,booktabs,lineno}
\usepackage[a4paper]{geometry}
\usepackage[parfill]{parskip}
\usepackage[usenames,dvipsnames]{color}
\modulolinenumbers[5]
\usepackage[font=footnotesize,labelfont=bf,margin=1cm]{caption} % caption formatting
\usepackage{setspace}
\usepackage{gensymb}
\usepackage{textcomp}
\usepackage{color} 
\usepackage{pbox}
\usepackage{float}
\usepackage[affil-it]{authblk} 
\usepackage{etoolbox}
\usepackage{lmodern}
\usepackage{rotating}
\usepackage[scientific-notation=true]{siunitx}


%\pdfminorversion=3

\renewcommand\Authfont{\fontsize{11}{12}\selectfont}
\renewcommand\Affilfont{\fontsize{8}{8}\itshape}

\doublespacing


\title{The difficult but important paths toward understanding variation in the amount of leaf area in vegetation}

\author[1]{
William K. Cornwell
}

\date{}

\affil[1] {Ecology and Evolution Research Centre, School of Biological, Earth and Environmental Sciences, University of New South Wales, Sydney, NSW 2052, Australia. Email for correspondence: \texttt{wcornwell@gmail.com}\\}


\begin{document}
	
\maketitle

\newpage	
	
	\parindent=1.5em
	\addtolength{\parskip}{.3em}
	\vfill
	
	\newpage
	
	The amount of leaf area per ground area (leaf area index, LAI) in different vegetation types has long been recognised as a key variable in vegetation ecology.  Very early in the development of plant ecology it became clear that the amount and distribution of leaf area influences both the production at the stand level and regeneration.  However, despite its clear ecological importance and decades of research, LAI is still not easy or simple to quantify.  \citet{braun1932plant} proposed a semi-quantitative method for estimating cover.  In this issue, \cite{dobert2015can} build on the  \citet{braun1932plant} method, developing a relationship between it and a destructive measure of LAI in the understory of a tropical forest, connecting a classic approach to a modern problem.  To understand where this fits in the literature it is worth considering exactly why LAI is crucial for vegetation ecology.  
	
	In an extremely prescient paper, \citet{monsi1953uber} laid out a number of ideas that have since become key parts of modern vegetation ecology, including a  quantitative understanding of leaf area index and its effect on light penetration through the canopy. (Note: this paper was originally published in German in the Japanese Journal of Botany and translated into English by Marcus Schortemeyer with the translation published in Annals of Botany in 2005.)  Considering the energetics of the system as a whole, one of the key observations of \citet{monsi1953uber} was that the dynamics at the bottom of the understory are inextricably linked to those at the top---in the darkest of the Japenese systems they studied, the bamboo forests, the understory species were becoming light--starved and etiolated. This is in some ways a special case of the regeneration niche \emph{sensu} \citet{grubb1977maintenance}, with leaf area and light penetration playing a key role \citep[see also][]{valladares2008shade}.  This separation of regeneration niche along a light/LAI gradient seems to even be true for groups of closely related species \citep[e.g. Hawaiian lobeliads, see][]{givnish2004adaptive}.  Thus, a full understanding of both community composition, stand dynamics, and evolution of photosynthetic traits rests on an understanding of the dynamics of leaf area index.  
	
    LAI at the stand level is, with few exceptions, an emergent feature of multiple species and multiple functional groups.  Some species make much higher LAI canopy compared to co-occurring species  \citep{kassnacht1997interrelationships}.   (And light penetration at a given LAI also varies among species.)  If the tendency to make a dense, high LAI canopy is also associated with the ability to regenerate at very low light conditions, this can, in certain conditions, represent a mechanism of positive density dependence, which may also be called ``niche construction'' \citep[\emph{sensu}][]{odling2003niche}.  However, especially in diverse systems it is rarely that simple: there is often a large suite of shade--tolerant species all seeking to regenerate in the limited opportunities \citep{valladares2008shade}.  
	
	Because the area of leaves affect fluxes of both water and carbon, there is also a key role for LAI in certain implementations of dynamic global vegetation models \citep[DGVMs, e.g.][]{murray2013evaluation}.  The variety of ways that DGVMs use or simulate LAI are beyond the scope of this short discussion piece, but suffice to say that better empirical datasets of LAI will certainly help those models.  LAI is also a key variable that is used to distinguish among early and late successional forests and among ecosystems for conservation purposes.  
	
	Given the utility of LAI for community, ecosystem, and global processes, there has been a great deal of interest in generating better spatial and temporal datasets.  However, there is not one agreed upon method: there are a variety of approaches from both the ground and remote sensing \citep{jonckheere2004review, breda2003ground} and a significant hurdle in temporal variability, both seasonal and successional.  Remote sensing products now exist for LAI at the global scale, but the training datasets are largely in North America and so its application to other regions introduces a large amount of error \citep[e.g. for Australia see][]{hill2006assessment}.  Moreover, the accuracy in other regions has also not been well tested, especially in areas were leaf area responds strongly to recent precipitation (or lack thereof).  Clearly the next generation of remotely sensed LAI products could use more geographically extensive and diverse training data.  
	
 If the techniques presented by \cite{dobert2015can} are further developed, expanded to include the canopy, there may be a way forward to more extensive---in both space and time---datasets of LAI.  Importantly, different approaches work better in different ecosystems---savannahs and woodlands, for example, require a different toolkits than forests \citep{fuentes2008automated, sea2011documenting}.  Fully understanding empirical variation in LAI will require a diverse array of ground and remote approaches.  
	
	
	\bibliographystyle{jecol}
	\bibliography{lai_comment.bib}

\end{document}